\titledquestion{Minimum Refueling}

A vehicle is driving from city A to city B on a highway.
The distance between A and B is $d$ kilometers.
The vehicle departs with $f_0$ units of fuel.
Each unit of fuel makes the vehicle travel one kilometer.
There are $n$ gas stations along the way.
Station $i$, denoted as $S_i$, is situated $p_i$ kilometers away from city A\@.
If the vehicle chooses to refuel at $S_i$, $f_i$ units would be added to the fuel tank whose capacity is unlimited.

Your job is to design a \textbf{greedy} algorithm that returns \textbf{the minimum number of refueling} to make sure the vehicle reaches the destination.

For example, if $d=100, f_0=20, (f_1,f_2,f_3,f_4)=(30,60,10,20), (p_1,p_2,p_3,p_4)=(10,20,30,60)$, the algorithm should return $2$.
There are two ways of refueling $2$ times.
The first one is:
\begin{itemize}
    \item Start from city A with $20$ units of fuel.
    \item Drive to station $1$ with $10$ units of fuel left, and refuel to $40$ units.
    \item Drive to station $2$ with $30$ units of fuel left, and refuel to $90$ units.
    \item Drive to city B with $10$ units of fuel left.
\end{itemize}
And the second one is:
\begin{itemize}
    \item Start from city A with $20$ units of fuel.
    \item Drive to station $2$ with $0$ units of fuel left, and refuel to $60$ units.
    \item Drive to station $4$ with $20$ units of fuel left, and refuel to $40$ units.
    \item Drive to city B with $0$ units of fuel left.
\end{itemize}

Note that:
\begin{enumerate}
    \item[1.] $0<p_1<p_2<\cdots<p_n<d$, and $\forall i\in[0,n],f_i\ge 1$.
    \item[2.] If the vehicle cannot reach the target, return \texttt{-1}.
    \item[3.] The time complexity of your algorithm should be $O(n\log n)$.
        You should use a max heap, and simply write \lstinline{heap.push(var)}, \lstinline{var = heap.pop()} in your \textbf{pseudocode}.
\end{enumerate}

\begin{parts}
    \part[2] Define the sub-problem $g(i)$ as the indices of an $n$-choose-$i$ permutation of stations $P=(P_1,P_2,\cdots,P_i)\in\text{Per}(n,i)$ satisfying the following conditions:
    \begin{gather*}
        g(i)=\{P\in\text{Per}(n,i):\forall k\in[1,n], (P_1,P_2,\cdots, P_k)\in\mathop{\arg\max}_{Q\in C_k}\sum_{j=1}^{i}f_{Q_j}\}\\
        C_k=\left\{Q\in\text{Per}(n,k):\forall l\in[1,k],f_0+\sum_{j=1}^{l-1}f_{Q_j}\ge p_{Q_l}\right\}\\
    \end{gather*}
    Then what is $g(1)$ and $g(2)$ in the example above?

    \part[2] How do you find one of the solutions in $g(i+1)$ by using one of the solutions in $g(i)$?
    And when does $g(i+1)=\emptyset$?

    \part[2] Prove the correctness of solving $g(i+1)$ by calling $g(i)$ when $g(i+1)\ne\emptyset$.

    \part[2] Define the problem $h(i)$ as the maximal distance that the vehicle can drive from city A:
    \begin{gather*}
        h(i)=f_0+\max_{Q\in E_i\cap C_i}\sum_{j=1}^{i}f_{Q_j}\\
        E_i=\left\{Q\in\text{Per}(n,i):Q_1<Q_2<\cdots<Q_i\right\}\\
        C_i=\left\{Q\in\text{Per}(n,i):\forall l\in[1,i],f_0+\sum_{j=1}^{l-1}f_{Q_j}\ge p_{Q_l}\right\}\\
    \end{gather*}
    Prove that $\forall P\in g(i), f_0+\sum_{j=1}^{i}f_{P_j}=h(i)$.

    \part[3] What is the relationship between $h(i)$ and the minimum number of refueling?
    And under what condition does the vehicle cannot reach the target?
    Based on your analysis above, describe your algorithm in \textbf{pseudocode}.

    \part[2] Analyse the time complexity based on your \textbf{pseudocode}.
\end{parts}

\begin{solution}
    \begin{enumerate}
        \item g(1) = {(2)} and g(2) = {(2,1)}
        \item For every set in g(i), choose the gas station that the vehicle haven't reached with max $f_j$.
              If the vehicle cannot reach that gas station $P_j$, which in other hand $P_j$ does not belong to $C_k$, we find the second max $f_j$.
              If the vehicle can reach, we push it back to this set.
              Repeat these operations, if in all sets all gas stations don't fit the requirement above, g(i+1) = $\varnothing$.
        \item For every solution from g(i) to g(i+1), we choose a gas station which can be reached and it's fuel is the max, thus it's obvious that if g(i+1) $\ne \varnothing$ solving g(i+1) by g(i) is correct.
        \item Knowing that each unit of fuel drives the vehicle one km, h(i) needs to choose the gas station that can reach with max $f_j$.
              g(i) choose the gas station that can reach with max $f_j$. It's obvious that g(i) and h(i) does the same operations.
              h(i) contains $f_0$ while g(i) doesn't. Thus the equation is correct.
        \item If h(i) is larger than or equal to the distance between two cities, then return i.
              If for all i, h(i) is less than the distance, then the vehicle cannot reach.
              \newpage
              \textbf{Pseudocode:}\\
              S is an array of the distances between the city A and gas stations.
              F is an array of the amount of fuel the gas stations can give.
              V is a vector which restore S and F in pair, like {$S_i,F_i$} and it is sorted according to $f_j$ in descending order.
              \begin{algorithm}[H]
                  \color{blue}
                  \begin{algorithmic}[1]
                      \Function{minimum\_fueling}{V}
                      \State answer $\gets$ ($f_0$,0)
                      \State goal $\gets$ (distance,0)
                      \State maxheap.push(goal)
                      \For{int i=0;i$<$V.size();++i}
                      \If{answer[0] $>$ V[i][0]}
                      \State answer[0] += V[i][0]
                      \State answer[1]++
                      \EndIf
                      \State maxheap.push(answer)
                      \If{maxheap[0][1] != 0}
                      \Return
                      \EndIf
                      \EndFor
                      \EndFunction
                  \end{algorithmic}
              \end{algorithm}
        \item In my pseudocode, the construction of max heap and the sort of vector is both $\theta(nlogn)$, and the loop will  run n times with $\theta(logn)$ operation.
              Thus the whole time complexity is $\theta(nlogn)$.
    \end{enumerate}
\end{solution}