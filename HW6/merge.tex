\titledquestion{K-Merge}

Recall that in merge sort, we learned how to merge two sorted arrays into one in linear time. In this question, we want to design a function that merge $K$ sorted arrays into one.

For example, here are 3 sorted arrays:
\begin{align*}
     & 1,5,9   \\
     & 2,3,6,7 \\
     & 4,6,7,9
\end{align*}
and we want to merge them as:
\begin{align*}
     & 1,2,3,4,5,6,6,7,7,9,9
\end{align*}

In the following question, suppose the sum of the lengths of the $K$ arrys is $n$. Here, assume $K=\omega(1)$ and $\log K=o(\log n)$.

\begin{parts}
    \part[2] Alice does not merge $K$ arrays at once. Instead, she decides to merge 2 of them each time. If she randomly choose 2 arrays and merges them, what is the time complexity of her algorithm in the \textbf{worst case}? You don't need to justify your answer.
    \begin{solution}
        The time complexity is $\theta(Kn)$.
    \end{solution}

    \part[2] Recall that in \textbf{worst case} merging two sorted arrays with lengths $n_1$ and $n_2$ needs $n_1+n_2-1$ comparisons. Now Alice wants to minimize the worst case number of comparisons in her algorithm. How should she choose the two arrays each time? Which algorithm does this strategy coincides with? \textbf{Briefly} give your answer.
    \begin{solution}
        %\vspace{2cm}
        To minimize the times of comparisons, Alice can first merge two arrays with lengths
        which are relatively less. Since the worst case is hard to be avoided, we should
        minimize the number of $n_1$ and $n_2$. If Alice merge arrays from those with longer
        lengths, these lengths will be added for more times. Thus Alice should start with arrays
        with less lengths. \\
        This strategy coincides with Huffman coding. Huffman coding lets words which are used
        more have less space. In this case space is confirmed while which array will be chosen
        for more times is the major problem.
    \end{solution}

    \part[2] Bob designs an algorithm that merges the $K$ arrays at once by modifying the merge function in merge sort. Each time, he looks up to the front element in each array and finds the smallest one among them. Then he puts this element at the back of his answer array and pop it from its original array. What is the time complexity of Bob's algorithm? \textbf{Briefly} justify your answer.
    \begin{solution}
        %\vspace{4cm}
        We should find the minimun among $K$ numbers, and repeat that operation. The operation
        will be repeated for no more than $n$ times though the last few operations will find minimum
        among less numbers. So that the time complexity is $O(Kn)$.
    \end{solution}

    \part[3] Now you need to improve Bob's algorithm to a better time complexity. \textbf{Briefly} describe your algorithm in natural language and give the complexity of your algorithm. Please focus on how to find the smallest front element in a shorter time.
    \begin{solution}
        %\vspace{7cm}
        we can first construct a min-heap to find the minimum in $K$ numbers. Then we pop the
        root of the heap, and insert a new element which is the next one in original array,
        thus make the operations after the first find be $\theta(log(K))$. The time complexity
        of constructing a min-heap is $\theta(K)$. For the whole progress, the time complexity
        will be reduced to $\theta(log(K)n)$.
    \end{solution}

\end{parts}

