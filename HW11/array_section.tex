\titledquestion{Array Section: Maximal Power}

Given a sequence of positive integers $A=\langle a_1,\cdots,a_n\rangle$, we want to divide it into several consecutive sections so that the sum of power of these sections
is maximized.

The power of section $\langle a_i,\cdots,a_j\rangle$ is defined as $aX_{i,j}^2+bX_{i,j}+c$, where
\begin{itemize}
    \item $a,b,c$ are given constants.
    \item $X_{i,j}=\sum\limits_{k=i}^j a_k$ is the sum of values in the section.
\end{itemize}

Please design a \textbf{dynamic programming} algorithm that returns the maximal sum of power of the sections that you divided.

For example, if \(a=-1,b=10,c=20\) and \(A=\langle 2,2,3,4\rangle\), the maximal sum of power is \(9\), and the way to divide the sequence is: \(\langle 2,2\rangle,\langle 3\rangle,\langle 4\rangle\).

\begin{parts}
    \part[3] Define the subproblems for $i\in[0,n]$: $OPT(i)=$ the maximal sum of power of if you only consider dividing the first $i$ elements. Give your Bellman equation to solve the subproblems.

    \begin{solution}
        \[
            OPT(i)=
            \begin{cases}
                0                                                       & \case{i=0} \\
                \max \{OPT(j-1)+aX_{i,j}^2+bX_{i,j}+c (1 \le j \le i)\} & \otherwise
            \end{cases}
        \]
    \end{solution}

    \part[1] What is the answer to this question in terms of $OPT$?

    \begin{solution}
        The answer is OPT(n).
    \end{solution}

    \part[3] What is the runtime complexity of your algorithm? (answer in $\Theta(\cdot)$)

    \textbf{Hint:} How will you compute $X_{j,i}$ in $\Theta(1)$ time? Please give your analysis of the preprocessing and computing complexity.

    \begin{solution} \\
        In this problem, we traverse every elements in A and then traverse all front elements.
        Thus we have $\Theta(n^2)$ times loop. \\
        In the loop, we compute $aX_{i,j}^2+bX_{i,j}+c$, which can be dealed in $\Theta(1)$ times. \\
        Thus the whole time complexity is $\Theta(n^2)$. \\
        We can use two variables to restore $X_{1,i}$ and compute them in outer loop. Call it sum1 and sum2.
        sum1 is used to restore $X_{1,i}$ and sum2 is used to compute $X_{j,i}$.
        Let sum2 = sum1 in the outer loop.
        In the inner loop, since j is started from 1, we can compute $X_{j,i}$ by $sum2 - a_j$.
        Thus the runtime will be $\Theta(1)$.
    \end{solution}

\end{parts}