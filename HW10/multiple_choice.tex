\titledquestion{Multiple Choices}

Each question has \textbf{one or more} correct answer(s). Select all the correct answer(s). For each question, you will get 0 points if you select one or more wrong answers, but you will get 1 point if you select a non-empty subset of the correct answers.

Write your answers in the following table.

%%%%%%%%%%%%%%%%%%%%%%%%%%%%%%%%%%%%%%%%%%%%%%%%%%%%%%%%%%%%%%%%%%%%%%%%%%%
% Note: The `LaTeX' way to answer a multiple-choices question is to replace `\choice'
% with `\CorrectChoice', as what you did in the first question. However, there are still
% many students who would like to handwrite their homework. To make TA's work easier,
% you have to fill your selected choices in the table below, no matter whether you use 
% LaTeX or not.
%%%%%%%%%%%%%%%%%%%%%%%%%%%%%%%%%%%%%%%%%%%%%%%%%%%%%%%%%%%%%%%%%%%%%%%%%%%

\begin{table}[htbp]
    \centering
    \begin{tabular}{|p{2cm}|p{2cm}|p{2cm}|}
        \hline
        (a) & (b) & (c)                \\
        \hline
        %%%%%%%%%%%%%%%%%%%%%%%%%%%%%%%%%%%%%%%%%%%%%%%%%%%%%%%%%%
        % YOUR ANSWER HERE.
        ABD & ACD & ABC \vspace{0.4cm} \\
        %%%%%%%%%%%%%%%%%%%%%%%%%%%%%%%%%%%%%%%%%%%%%%%%%%%%%%%%%%
        \hline
    \end{tabular}
\end{table}

\begin{parts}

    \part[2] Which of the following statements about Dijkstra's algorithm is/are true?

    \begin{choices}
        \CorrectChoice Once a vertex is marked as visited, its distance will never be updated.
        \CorrectChoice The time complexity of Dijkstra's algorithm using complete binary heap is $\Theta(|E|\log |V|)$.
        \choice If we use Dijkstra's algorithm to find the distance from vertex $s$ to vertex $t$, then when we first push $t$ into the heap, we find the shortest path from $s$ to $t$ and stop the algorithm.
        \CorrectChoice If vertex $u$ is marked visited before $v$, then $\texttt{dist[u]}\le \texttt{dist[v]}$.
    \end{choices}

    \part[2] Which of the following statements about A* search algorithm is/are true?

    \begin{choices}
        \CorrectChoice If we use heuristic function $h(u)=c$ for any $u\in V$ where $c$ is a positive constant, then the A* search algorithm will be the same with Dijkstra's algorithm.
        \choice An admissible heuristic function ensures optimality of both A* tree search algorithm and A* graph search algorithm.
        \CorrectChoice A consistent heuristic function ensures optimality of both A* tree search algorithm and A* graph search algorithm.
        \CorrectChoice Suppose we want to search for the shortest path from a city to another on a map. If we use the heuristic function $h(u)=dis(u,t)$, the Euclidean distance between $u$ and the destination $t$, then this is a consistent heuristic function.
    \end{choices}

    \part[2] Which of the following statements about Bellman-Ford algorithm is/are true?

    \begin{choices}
        \CorrectChoice In a DAG with probably negative edge weights, Bellman-Ford algorithm is guaranteed to find the shortest path from source $s$ to any vertex if it can be reached from $s$.
        \CorrectChoice Suppose the unique shortest path from source $s$ to a vertex $t$ has $l$ edges. It is impossible that we find this shortest path from $s$ to $t$ in less than $l$ iterations.
        \CorrectChoice If during the $i$-th iteration, there is no update on \ttt{dist} array, then we can stop the algorithm but still get correct results.
        \choice After Bellman-Ford algorithm, the \ttt{prev} array defines a tree rooted at the source vertex, and the tree is also an MST of the original graph.
    \end{choices}



\end{parts}

