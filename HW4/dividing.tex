\titledquestion{Dividing with Creativity}

In this question, you are required analyze the run-time of algorithms with different dividing methods mentioned below. For each subpart except the third one, your answer should include:
\begin{enumerate}
    \item Describing the recurrence relation of the run-time $T(n)$. (Worth 1 point in 4)
    \item Finding the asymptotic order of the growth of $T(n)$ i.e. find a function $g$ such that $T(n) = O(g(n))$. Make sure your upper bound for $T(n)$ is tight enough. (Worth 1 point in 4)
    \item Show your \textbf{reasoning} for the upper bound of $T(n)$ or your process of obtaining the upper bound starting from the recurrence relation step by step. (Worth 2 points in 4)
\end{enumerate}

In each subpart, you may ignore any issue arising from whether a number is an integer as well as assuming \(T(0) = 0\) and \(T(1) = 1\). You can make use of the Master Theorem, Recursion Tree or other reasonable approaches to solve the following recurrence relations.

\begin{parts}
    \part[4] An algorithm $\mathcal{A}_1$ takes $\Theta(n)$ time to partition the original problem into 2 sub-problems, one of size $\lambda n$ and the other of size $(1-\lambda)n$ (here $\lambda \in \left(0, \frac{1}{2}\right)$), then recursively runs itself on both of the 2 sub-problems and finally takes $\Theta(n)$ time to merge the answers of the 2 sub-problems.

    \begin{solution}
        %%%%%%%%%%%%%%%%%%%%%%%%%%%%%%%%%%%%%%%%%%%%%%%%%
        % Replace `\vspace{5in}' with your answer.
        %\vspace{5in}
        \begin{enumerate}
            \item $T(n) = T(\lambda n) + T((1-\lambda)n) + \theta(n)$
            \item g(n) = nlogn
            \item Knowing that
                  \begin{equation}
                      \begin{array}{l}
                          T(n) = T(\lambda n) + T((1-\lambda)n) + \theta(n)                                     \\
                          T(\lambda n) = T(\lambda^2 n) + T(\lambda (1-\lambda)n) + \theta(\lambda n)           \\
                          T((1-\lambda)n) = T((1-\lambda)\lambda n) + T((1-\lambda)^2 n) + \theta((1-\lambda)n) \\
                          T(n) = T(\lambda^2 n) + 2T(\lambda (1-\lambda)n) + T((1-\lambda)^2 n) + \theta(2n)
                      \end{array}
                  \end{equation}
                  Thus assuming that the recrusion goes k times, the whole time complexity is $\theta(kn)$. \\
                  Repeat the operations above until $(1-\lambda)^k n \approx 1$, at which time other items $\approx 0$.
                  So that $k = \theta(logn)$, $g(n) = \theta(nlogn)$.
        \end{enumerate}
        %%%%%%%%%%%%%%%%%%%%%%%%%%%%%%%%%%%%%%%%%%%%%%%%%    
    \end{solution}

    \newpage

    \part[4] An algorithm $\mathcal{A}_2$ takes $\Theta(n)$ time to partition the original problem into 2 sub-problems, one of size $k$ and the other of size $(n - k)$ (here $k \in \mathbb{Z}^+$ is a constant), then recursively runs itself on both of the 2 sub-problems and finally takes $\Theta(n)$ time to merge the answers of the 2 sub-problems.

    \begin{solution}\\
        %%%%%%%%%%%%%%%%%%%%%%%%%%%%%%%%%%%%%%%%%%%%%%%%%
        % Replace `\vspace{1.5in}' with your answer.
        %\vspace{1.5in}
        \begin{enumerate}
            \item $T(n) = T(k) + T(n-k) + \theta(n)$
            \item g(n) = $n^2$
            \item
                  \begin{equation}
                      \begin{array}{l}
                          T(n) = T(k) + T(n-k) + \theta(n)                  \\
                          T(k) = T(0) + T(k) + \theta(1) = T(k) + \theta(1) \\
                          T(n-k) = T(k) + T(n-2k) + \theta(n)               \\
                          T(n) = 2T(k) + T(n-2k) + \theta(2n)
                      \end{array}
                  \end{equation}
                  Thus we have $T(n) = \frac{n}{k}T(k) + \theta(\frac{n^2}{k})$. The time complexity is $\theta(n^2)$.
        \end{enumerate}
        %%%%%%%%%%%%%%%%%%%%%%%%%%%%%%%%%%%%%%%%%%%%%%%%%    
    \end{solution}

    \part{} Solve the recurrence relation $T(n) = T(\alpha n) + T(\beta n) + \Theta(n)$ where $\alpha + \beta < 1$ and $\alpha \geq \beta$.
    \begin{subparts}
        \subpart[2] Fill in the \textbf{four} blanks in the mathematical derivation snippet below.
        \begin{align*}
            T(n) & = T(\alpha n) + T(\beta n) + \Theta(n)                                                                         \\
                 & = (T(\alpha ^2 n) + T(\alpha \beta n) + \Theta(\alpha n)) +
            (T(\alpha \beta n) + T(\beta ^2 n) + \Theta(\beta n)) + \Theta(n)                                                     \\
                 & = (T(\alpha ^2 n) + 2T(\alpha \beta n) + T(\beta ^2 n)) + \Theta(n) (1 + (\alpha + \beta))                     \\
                 & = \dots                                                                                                        \\
                 & = \sum _ {i=0} ^k \binom{k}{i} T(\alpha^i \beta^{k-i} n) + \Theta(n) \sum _ {j = 0} ^ {k} {(\alpha + \beta)^j}
        \end{align*}

        \subpart[3] Based on the previous part, complete this question.
        \begin{solution} \\
            %%%%%%%%%%%%%%%%%%%%%%%%%%%%%%%%%%%%%%%%%%%%%%%%%
            % Replace `\vspace{3in}' with your answer.
            %\vspace{3in}
            Knowing $\alpha \ge \beta$, make $k = log_{\alpha}{n}$. Thus the equation will be
            \begin{equation}
                \begin{aligned}
                    T(n) = & \theta(n) \sum_{j=0}^{log(n)} (\alpha + \beta)^j
                    =      & \theta(n)
                \end{aligned}
            \end{equation}
            %%%%%%%%%%%%%%%%%%%%%%%%%%%%%%%%%%%%%%%%%%%%%%%%%
        \end{solution}
    \end{subparts}

    \newpage

    \part[4] An algorithm $\mathcal{A}_3$ takes $\Theta(\log n)$ time to convert the original problem into 2 sub-problems, each one of size $\sqrt{n}$, then recursively runs itself on both of the 2 sub-problems and finally takes $\Theta(\log n)$ time to merge the answers of the 2 sub-problems.

    Hint: W.L.O.G., you may assume $n = 2^m$ for $m \in \mathbb{Z}$.
    \begin{solution} \\
        %%%%%%%%%%%%%%%%%%%%%%%%%%%%%%%%%%%%%%%%%%%%%%%%%
        % Replace `\vspace{3in}' with your answer.
        %\vspace{3in}
        From the problem we have
        \begin{equation}
            T(n) = 2T(\sqrt{n}) + \theta(log(n))
        \end{equation}
        Let $n = 2^m$, we have $a = 2$, $b = 2^{\frac{m}{2}}$, f(n) = m. Since
        \begin{equation}
            n^{log_{b}^{a}} = (2^m)^{\frac{2}{m}} = 4 < n^d = m
        \end{equation}
        T(n) = $\theta(log(n))$.
        %%%%%%%%%%%%%%%%%%%%%%%%%%%%%%%%%%%%%%%%%%%%%%%%%    
    \end{solution}
\end{parts}