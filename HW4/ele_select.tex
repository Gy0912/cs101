\titledquestion{Element(s) Selection}

\begin{parts}
  \part{} \textbf{Selection of the \(k\)-th Minimal Value} \par
  In this part, we will design an algorithm to find the \(k\)-th minimal value of a given array \(\langle a_1,\cdots,a_n\rangle\) of length \(n\) with \emph{distinct} elements for an integer \(k\in[1,n]\). We say \(a_x\) is the \(k\)-th minimal value of \(a\) if there are exactly \(k-1\) elements in \(a\) that are less than \(a_x\), i.e.
  \[\left|\left\{i\mid a_i<a_x\right\}\right|=k-1.\]
  Consider making use of the `\textbf{partition}' procedure in quick-sort. The function has the signature
  \begin{cpp}
    int partition(int a[], int l, int r);
  \end{cpp}
  which processes the subarray \(\langle a_l,\cdots,a_r\rangle\). It will choose a pivot from the subarray, place all the elements that are less than the pivot before it, and place all the elements that are greater than the pivot after it. After that, the index of the pivot is returned.

  Our algorithm to find the \(k\)-th minimal value is implemented below.
  \begin{cpp}
    // returns the k-th minimal value in the subarray a[l],...,a[r].
    int kth_min(int a[], int l, int r, int k) {
        auto pos = partition(a, l, r), num = pos - l + 1;
        if (num == k)
        return a[pos];
        else if (num > k)
        return kth_min(a, l, pos - 1, k);
        else
        return kth_min(a, pos + 1, r, k - num);
      }
  \end{cpp}
  By calling \lstinline{kth_min(a, 1, n, k)}, we will get the answer.

  \begin{subparts}
    \subpart[2] Fill in the blanks in the code snippet above.
    \subpart[2] What's the time complexity of our algorithm in the \textbf{worst case}? Please answer in the form of \(\Theta(\cdot)\) and fully justify your answer.
    \begin{solution} \\
      %%%%%%%%%%%%%%%%%%%%%%%%%%%%%%%%%%%%%%%%%%%%%%%%%
      % Replace `\vspace{2.5in}' with your answer.
      %\vspace{2.5in}
      In the worst case, we can make every pivot be the minimum and the number we tend to find be the maximum.
      Thus in every recrusion the program can only remove 1 number. To find the final answer, the function $kth_min$
      will be operated for n times. In the function partition, the worst case will be the minimum we chosen is
      at the end of array, thus we need to swap $\frac{n}{2}$ times. So that the complexity for k-length array
      is $\theta(k)$. Then the whole complexity is $\sum_{k=1}^{n}k$, which is $\theta(n^2)$.
      %%%%%%%%%%%%%%%%%%%%%%%%%%%%%%%%%%%%%%%%%%%%%%%%%
    \end{solution}
  \end{subparts}

  \newpage

  \part{} \textbf{Batched Selection} \par

  Despite the worse-case time complexity of the algorithm in part(a), it actually finds the $k$-th minimal value of \(\langle a_1,\cdots,a_n\rangle\) in expected $O(n)$ time. In this part, we will design a divide-and-conquer algorithm to answer $m$ selection queries for distinct $k_1, k_2, \cdots, k_m$ where $k_1 < k_2 < \cdots < k_m$ on an given array $a$ of n distinct integers (i.e. finding the $k_1$-th, $k_2$-th,$\cdots$,$k_m$-th minimal elements of $a$) and here $m$ satisfies $m = \Theta(\log n)$.

  \begin{subparts}
    \subpart[1] Given that $x$ is the $k_p$-th minimal value of $a$ and $y$ is the $k_q$-th minimal value of $a$ for $1 \leq p < q \leq m$, which of the following is true?

    \begin{oneparcheckboxes}
      \CorrectChoice $x < y$
      \choice $x = y$
      \choice $x > y$
    \end{oneparcheckboxes}



    \subpart[2] Suppose by calling the algorithm in part(a), we have already found $z$ to be the $k_l$-th minimal value of $a$ for $1 < l < m$. Let $L = \left\{a_i \mid a_i < a_z\right\}$ and $R = \left\{a_i \mid a_i > a_z\right\}$. What can you claim about the $k_1$-th,$\cdots$,$k_{l-1}$-th minimal elements of $a$ and the $k_{l+1}$-th,$\cdots$,$k_{m}$-th minimal elements of $a$?

    \begin{solution} \\
      %%%%%%%%%%%%%%%%%%%%%%%%%%%%%%%%%%%%%%%%%%%%%%%%%
      % Replace `\vspace{1in}' with your answer.
      %\vspace{1in}
      we know that for the $k_1th$ to $k_{l-1}th$ minimal elements, they are less than $k_lth$ minimum, so that they belong
      to L. On the other hand, minimal elements that $k_{l+1}th$ to $k_mth$ are larger than $k_lth$ minimum, thus belong to R.
      %%%%%%%%%%%%%%%%%%%%%%%%%%%%%%%%%%%%%%%%%%%%%%%%%    
    \end{solution}

    \subpart[6] Based on your answers of previous parts, design a divide-and-conquer algorithm, \textbf{which calls the algorithm in part(a) as a subroutine}, for this problem. Your algorithm should runs in \textbf{expected} $O(n \log m) = O(n \log \log n)$ time. Any algorithms that run in $\Omega(n \log n)$ time will get no credit. Make sure to provide \textbf{clear description} of your algorithm design in \textbf{natural language}, with \textbf{pseudocode} if necessary.

    \begin{solution} \\
      %%%%%%%%%%%%%%%%%%%%%%%%%%%%%%%%%%%%%%%%%%%%%%%%%
      % Replace `\vspace{2.5in}' with your answer.
      %\vspace{2.5in}
      \textbf{Algorithm Design:}
      \begin{enumerate}
        \item Sort m minimums into descending order, using quick sort or merge sort.
              Assume we use array A to store these minimums. The whole array called R.
        \item Choose the median element of A and find its location in R, using some part of the code in part(a). Return the position of the madian element.
              Divide the R and A into two parts, with one all number is larger than median number and the other is less than median number.
        \item For the divided R and A, called A' and R', choose the median number of A'. Find its location in R'. Return its location and divide A' and R' in the same way.
        \item Repeat the operations above until all elements are found.
      \end{enumerate}
      \textbf{Pseudocode:}\\
      $A$ is an array to restore the numbers to find. $left$ and $right$ are indices of the leftmost and rightmost elements in array $A$.
      $k$ means the length of A. $R$ is an array for all n numbers.
      $head$ is the head address of whole n numbers. $tail$ is the tail address of whole n numbers.
      In this pseudocode, some functions like $Quick_sort$, which is not changed, or partition,
      which is a part of functions that are not changed, will be used without concrete realization.
      \begin{algorithm}[H]
        \color{blue}
        \begin{algorithmic}[1]
          \Function {$quick\_sort$}{A, left, right}
          \EndFunction

          \Function {partition}{R, head, tail}
          \EndFunction

          \Function{$kth\_min$}{R, head, tail, k}
          \State pos = partition(R, head, tail) \\
          \State num = $pos - head + 1$ \\
          \If {num == k}
          \State \If {k != 1}
          \State \State \Return $kth\_min$(R, head, $pos - 1$, k/2)
          \State \State \Return $kth\_min$(R, $pos + 1$, tail, k/2)
          \State \Else
          \State \Return pos
          \State \EndIf
          \ElsIf {num $>$ k}
          \State \Return $kth\_min$(R, head, $pos - 1$, k)
          \Else
          \State \Return $kth\_min$(R, pos $+$ 1, tail, $k - num$)
          \EndIf
          \EndFunction
        \end{algorithmic}
      \end{algorithm}
      %%%%%%%%%%%%%%%%%%%%%%%%%%%%%%%%%%%%%%%%%%%%%%%%%    
    \end{solution}

    \newpage

    \subpart[2] Provide your reasoning for why your algorithm in the previous part runs in expected $O(n\log m)$ time using the \textbf{recursion-tree} method.
    \begin{solution} \\
      %%%%%%%%%%%%%%%%%%%%%%%%%%%%%%%%%%%%%%%%%%%%%%%%%
      % Replace `\vspace{3in}' with your answer.
      %\vspace{3in}
      For R with length of n and A with length of m, we have the complexity
      \begin{equation}
        \begin{array}{l}
          T(find) = \theta(n) \\
          T(partition) = \theta(n)
        \end{array}
      \end{equation}
      Thus for R of length n, the time complexity is $\theta(n)$. \\
      Then persue the height of the recrusion tree.
      We have to find m numbers, and everytime we divide the array R into 2 subarraies, so that we have $height = log(m)$. \\
      The recrusion will be
      \begin{equation}
        T(n) = 2T(\frac{n}{2}) + \theta(n)
      \end{equation}
      Since the recrusion end in log(m) times, the whole time complexity is $\theta(nlog(m))$.
      %%%%%%%%%%%%%%%%%%%%%%%%%%%%%%%%%%%%%%%%%%%%%%%%%      
    \end{solution}

  \end{subparts}

\end{parts}