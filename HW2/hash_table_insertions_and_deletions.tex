\titledquestion{Hash Table Insertions and Deletions}
Consider an empty hash table of capacity 7 and with hash function \(h(k) = (3k + 6) \mod 7\). Collisions are resolved by quadratic probing with the probing function \(H_i(k) = (h(k) + i^2)\mod 7\), paired with lazy erasing. We will give three kinds of instructions, which are among the set  $\{$\ttt{Insert, Delete, Search}$\}$. For \texttt{Insert/Delete} instructions, you need to fill the hash table after each instruction. For \texttt{Search} instructions, write down probing sequence (index). Use `\(\ttt{D}\)' to indicate that the bin has been marked as deleted.
\newcolumntype{x}{>{\centering\arraybackslash\hspace{0pt}}p{0.5cm}}

\begin{parts}
  \part [1] \ttt{Insert 9}
  \begin{table}[hbt!]
    \centering
    \begin{tabular}{|c|x|x|x|x|x|x|x|}
      \hline
      Index     & 0 & 1 & 2 & 3 & 4 & 5 & 6 \\
      \hline
      Key Value &   &   &   &   &   & 9 &   \\
      \hline
    \end{tabular}
  \end{table}
  \part[1] \ttt{Insert 17}
  \begin{table}[hbt!]
    \centering
    \begin{tabular}{|c|x|x|x|x|x|x|x|}
      \hline
      Index     & 0 & 1  & 2 & 3 & 4 & 5 & 6 \\
      \hline
      Key Value &   & 17 &   &   &   & 9 &   \\
      \hline
    \end{tabular}
  \end{table}
  \part[1] \ttt{Insert 32}
  \begin{table}[hbt!]
    \centering
    \begin{tabular}{|c|x|x|x|x|x|x|x|}
      \hline
      Index     & 0 & 1  & 2 & 3 & 4  & 5 & 6 \\
      \hline
      Key Value &   & 17 &   &   & 32 & 9 &   \\
      \hline
    \end{tabular}
  \end{table}
  \part[1] \ttt{Insert 24}
  \begin{table}[hbt!]
    \centering
    \begin{tabular}{|c|x|x|x|x|x|x|x|}
      \hline
      Index     & 0 & 1  & 2  & 3 & 4  & 5 & 6 \\
      \hline
      Key Value &   & 17 & 24 &   & 32 & 9 &   \\
      \hline
    \end{tabular}
  \end{table}
  \part[1] \ttt{Insert 18}
  \begin{table}[hbt!]
    \centering
    \begin{tabular}{|c|x|x|x|x|x|x|x|}
      \hline
      Index     & 0 & 1  & 2  & 3 & 4  & 5 & 6  \\
      \hline
      Key Value &   & 17 & 24 &   & 32 & 9 & 18 \\
      \hline
    \end{tabular}
  \end{table}
  \part[1] \ttt{Search 18}
  \begin{solution}
    %\vspace{1in}
    First, we compute h(18), which is 4. Then, we find the index 4, only to find that there
    is a value in index 4 and the value is not equal to 18. Thus we compute $H_i(k) = (h(k) + i^2) \ mod \ 7$,
    we have index 5, which has a value. We repeat that way with increaing i, then go to index
    1$((4 + 2^2) \ mod \ 7 = 1)$, which has a value, then go to index 6$((4 + 3^2) \ mod \ 7 = 6)$,
    whose value is 18, thus finish searching.
  \end{solution}
  \part[1] \ttt{Delete 32}
  \begin{table}[hbt!]
    \centering
    \begin{tabular}{|c|x|x|x|x|x|x|x|}
      \hline
      Index     & 0 & 1  & 2  & 3 & 4 & 5 & 6  \\
      \hline
      Key Value &   & 17 & 24 &   & D & 9 & 18 \\
      \hline
    \end{tabular}
  \end{table}
  \newpage
  \part[1] \ttt{Insert 25}
  \begin{table}[hbt!]
    \centering
    \begin{tabular}{|c|x|x|x|x|x|x|x|}
      \hline
      Index     & 0 & 1  & 2  & 3 & 4  & 5 & 6  \\
      \hline
      Key Value &   & 17 & 24 &   & 25 & 9 & 18 \\
      \hline
    \end{tabular}
  \end{table}
  \part[3] Suppose that the collisions are resolved by linear probing.
  \begin{enumerate}[i.]
    \item Write down the content of the hash table after \ttt{Insert 9, 17, 32, 24, 18}.
          \begin{table}[hbt!]
            \centering
            \begin{tabular}{|c|x|x|x|x|x|x|x|}
              \hline
              Index     & 0 & 1  & 2  & 3 & 4  & 5 & 6  \\
              \hline
              Key Value &   & 17 & 24 &   & 32 & 9 & 18 \\
              \hline
            \end{tabular}
          \end{table}
    \item What is the load factor $\lambda$ ?
          \begin{solution}
            %\vspace{1in}
            In this hash table we have a capacity of 7 with 5 inserted elements, Thus the load
            factor is $\frac{5}{7}$.
          \end{solution}
  \end{enumerate}
\end{parts}
